\documentclass[11pt]{article}

\usepackage[utf8]{inputenc}
\usepackage[english]{babel}

\usepackage{listings}
\lstset{
  breaklines=true,
}
\usepackage{indentfirst}
\usepackage{natbib}
\usepackage{graphicx}

\begin{document}
\section{LogiQL encoding}
LogiQL is a declarative logic programming language ran by LogicBlox database system. Given a set of constraints from a particular type system, we can encode those constraints into LogiQL form, and solve them by LogicBlox.
\subsection{Basic Encoding} 
Basic encoding consists some predicates that don't depend on type systems.
\begin{lstlisting}
variable(v),hasvariableName(v:i)->int(i).
/*Variable is an entity type, and hasvariableName is a refmode predicate for it. A variable name is represented as a integer. */ 
\end{lstlisting}
This predicate stores all the variables(slotID-typeQualifier), which will be used for all other tables. 

\begin{lstlisting}
isAnnotated[v] = i -> variable(v), boolean(i).
/* */ 
\end{lstlisting}
This predicate represents whether the constraint variable v.slotID is annotated by v.typeQualifier.

\begin{lstlisting}
hasToBeTrue[v] = h -> variable(v), boolean(h).
\end{lstlisting}
This predicate is for the situation that a variable must be true, which is corresponding to a slot equals to a constant slot, or the equality constraint between a constant slot and a variable slot.
\begin{lstlisting}
set_1_variable[v] = s -> variable(v), boolean(s).
set_1_variable[v] = true <- hasToBeTrue[v] = true.
\end{lstlisting}
set\_x\_variable means a group of variable, x is the number of variables in this set, it would be true if one of variables in this set is true. The following set only contains one variable.following set only contains one variable.

\begin{lstlisting}
implies[v1,v2] = e -> variable(v1), variable(v2), boolean(e).
\end{lstlisting}
A table represents implies logic: "v1 --> v2"

\begin{lstlisting}
rightSide[v] = r -> variable(v), boolean(r).
\end{lstlisting}
rightSide table indicates whether the variable in right side is operated by logic "not"


 \par 

\end{document}
